\documentclass[a4paper, 12pt]{article}
\usepackage[left=2.5cm,right=2.5cm,top=2.5cm,bottom=2.5cm]{geometry}
\usepackage[utf8]{inputenc}
\usepackage[T1]{fontenc}
\usepackage[francais]{babel}
\usepackage{layout}
\usepackage{graphicx}
\usepackage{hyperref}

\author{Mlle Siyu HUANG}
\title{Projet Internet et Bases de données}
\date{17 Mai 2017}

\renewcommand{\baselinestretch}{1.5}
\begin{document}

\pagestyle{empty}
\maketitle
\pagebreak


\tableofcontents
\pagebreak

\section{Introduction}

Grâce à ce cours « Internet et Bases de données » dans ce semestre, nous avons appris à traiter les données, et nous avons organisé ces données par la langue PHP pour créer un site. Les personnes peuvent trouver ces données plus facile pour obtenir les résultats qu’ils veulent.\\

Ce que je fais est un site de l’hôtel pour chercher les informations de l’hôtel. Je suppose que quand les touristes veulent voyager pour profiter leur vacances, ils ont besoin d’information de l’hôtel d’arrivé, donc ce site peut leur aider trouver les hôtels dans la ville d’arrivée. Ceci est mon but.\\

\section{Explication et Processus}

Je choisis les données de l’hôtel sur la région Hauts- de-France où il y a 20 villes les plus peuplées, je trouve ces tables de donnée dans le site « www.data.gouv.fr ». D’abord, les visiteurs peuvent choisir la ville qu’ils veulent voyager, enfin mon site peut donner les résultats sur les hôtels de cette ville. Les informations de l’hôtel sont inclut : le classement de l’hôtel, le nom de l’hôtel, l’adresse, le courriel et le site web de l’hôtel. Ces sont les informations nécessaires que nous pouvons trouver dans mon site.\\

Tout d’abord, je suis déterminer mon sujet, alors je cherche et télécharge des données en ligne pour faire un tableau en csv. Après je utilise le Terminal du système Linux, le tableau en csv transformé en sqlite, le suffixe nom de fichier ‘.db’. Ensuite, j’écris deux fichier en PHP, un nom de fichier est ‘projet.php’, ce fichier a été produit une page Web pour rechercher, un autre nom de fichier est ‘resultat.php’, ce fichier a été mis à les données d’hôtel, puis construire la relation entre les deux fichier. Dans ces deux pages, j’ai aussi inséré des images pour avoir le plus beau site. Enfin, j’envois ces fichiers sur le site « github.com », voici le lien de mon projet :
\url{https://github.com/siyu113/Siyu-Huang-WEB17-L3Miashs/tree/L3Miashs} \\

\section{Conclusion}

Dans le processus que je fais ce projet, j’ai rencontré beaucoup de petits problèmes, mais finalement j’ai les résolu. Ce projet me donne maître plus qualifié sur la langue PHP, et une grande aide pour le filtrage des données et créer un site.\\



\end{document}